% Macros used in thesis

\newcommand{\E}{\mathbb{E}}
\newcommand{\given}{\, | \,}

\newcommand{\diag}[1]{\mathrm{diag}\left(#1\right)}
\newcommand{\tr}[1]{\mathrm{tr}\left(#1\right)}
\newcommand{\diagvec}[1]{\mathrm{\overrightarrow{diag}}\left(#1\right)}
\newcommand{\unitmat}[0]{\vec{1}\vec{1}^{\top}}
\DeclareMathOperator{\argmax}{arg\,max}
\DeclareMathOperator{\argmin}{arg\,min}
\newcommand{\dd}{\mathrm{d}}
\DeclareMathOperator{\Unif}{Unif}
\DeclareMathOperator{\Multinom}{Multinom}
\DeclareMathOperator{\Bernoulli}{Bern}
\DeclareMathOperator{\Expo}{Expo}
\DeclareMathOperator{\Poisson}{Poisson}
\DeclareMathOperator{\logit}{logit}
\DeclareMathOperator{\Var}{Var}
\DeclareMathOperator{\Cov}{Cov}
\DeclareMathOperator{\Median}{Med}
\newcommand{\range}[2]{\left\{#1 \ldots #2\right\}}
\newcommand{\fnDef}[3]{#1 \,:\, #2 \rightarrow #3}
\DeclareMathOperator{\sign}{sign}

\newcommand{\indep}[0]{\perp\!\!\!\perp}
%\def\indep{\perp}

\newtheorem{example}{Example}
\newtheorem{conjecture}{Conjecture}
\newtheorem{remark}{Remark}
\newtheorem{theorem}{Theorem}
\newtheorem{lemma}{Lemma}
\newtheorem{fact}{Fact}
\newtheorem{note}{Note}
\newtheorem{definition}{Definition}
\newtheorem{assumption}{Assumption}
\newtheorem{problem}{Problem}
\newtheorem{result}{Result}

% Macros for generic two-phase problem
\newcommand{\param}{\theta}
\newcommand{\pobs}{p_Y}
\newcommand{\psci}{p_X}
\newcommand{\pproc}{p_T}
\newcommand{\pwork}{\tilde{p}_X}
\newcommand{\pywork}{\tilde{p}_Y}
\newcommand{\obsparam}{\xi}
\newcommand{\workparam}{\eta}

% Macros for estimators
\newcommand{\est}{\hat{\param}} % Generic MLE
\newcommand{\multiestinf}{\bar{\param}_{\infty}} % MI asymptotic estimator

% Macros for procedures
\newcommand{\proc}{\mathcal{P}}
