%!TEX TS-program = xelatex
% !TEX root = ../thesis.tex
% Do not delete; used to set build system

\begin{savequote}[75mm] 
Nulla facilisi. In vel sem. Morbi id urna in diam dignissim feugiat. Proin molestie tortor eu velit. Aliquam erat volutpat. Nullam ultrices, diam tempus vulputate egestas, eros pede varius leo.
\qauthor{Quoteauthor Lastname} 
\end{savequote}

\chapter{Introduction}

\lettrine{A}{dvances} in technology have enabled an unprecedented growth in the depth and breadth of scientific data.
Much of this data is generated by experimental processes that create complex forms of measurement error and missing data.
This complexity makes statistical thinking and modeling necessary at all levels of processing and inference.
An increasing amount of statistical research involves inference conducted by multiple analysts or groups, with each using another's output in sequence.

Tackling these new challenges and volumes of data requires new approaches to statistical computing.
Data sizes and computational demands are now outpacing gains speed for individual CPUs.
Thus, we increasingly need to develop parallel and distributed approaches to large-scale inference.
This is difficult for many standard classes of statistical algorithms, such as MCMC and EM, due to their inherently sequential nature.
However, it is often possible to constructed efficient distributed algorithms by leveraging the conditional independence structure of statistical models, as we demonstrate in Chapters \ref{ch:nucleosomes} and \ref{ch:proteomics}.

\section{Principles for multiphase inference}




\section{Inferring nucleosome positions}



\section{Getting more from LC-MS/MS proteomics data}


